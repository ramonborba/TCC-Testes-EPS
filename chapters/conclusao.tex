% ----------------------------------------------------------
\chapter{Conclusão}
% ----------------------------------------------------------

Neste trabalho, foram realizadas análises acerca de deferentes topologias e arquiteturas de módulos de gerenciamento de energia de CubeSats, bem como de suas respectivas campanhas de testes. Foram analisadas também as normas da \gls{ECSS} relacionadas à testes e verificação, considerando o contexto de um módulo de CubeSat. Com isto foram identificadas as principais funcionalidades presentes nas diferentes arquiteturas, bem como os principais tipos de teste executados. Ainda, foram identificados os principais aspectos das normas em relação à elaboração de planos de teste, bem como discutidas as principais considerações a serem feitas ao aplicar-se estas normas a um módulo de CubeSat.

A partir destas análises, foi elaborado o documento \textit{EPS Test Plan Guidelines} contendo diretrizes e orientações acerca da elaboração de planos de testes para módulos \gls{EPS} de CubeSats, baseado nas normas da \gls{ECSS}, de forma que seja aplicável à diferentes topologias e arquiteturas.

Por fim, como forma de demonstração, um plano de testes, de forma simplificada, para a qualificação do módulo de gerenciamento de energia \gls{EPS2}, desenvolvido no SpaceLab, foi proposto, baseado nas diretrizes e orientações desenvolvidas neste trabalho.

O documento de diretrizes desenvolvido possibilitará a elaboração de planos de teste melhor estruturados, organizados e completos para os módulos \gls{EPS} desenvolvidos no SpaceLab.
Contribuindo de forma direta para o aumento da confiabilidade destes módulos e, por consequência, das missões do laboratório.
Espera-se também que outros grupos possam ser auxiliados por este trabalho, visto que o documento estará disponível de forma aberta.

Outro aspecto positivo deste trabalho é que, apesar do direcionamento para módulos \gls{EPS}, as discussões e considerações levantadas acerca da aplicação das normas e das adaptações para o contexto de um CubeSat podem facilmente ser estendidas aos outros módulos de serviço.
Ainda, a partir das análises realizadas, foi possível identificar diversos pontos passíveis de melhoria nos processos de verificação e testes adotados no SpaceLab, que podem vir a ser o foco de próximos trabalhos.

\section{Trabalhos Futuros}

Como sugestão para trabalhos futuros, a análise de um número maior de arquiteturas, bem como de campanhas de teste, poderá resultar na identificação de funcionalidades ainda não contempladas,contribuindo para a elaboração de uma matriz de testes base ainda mais abrangente.

Considerando as aplicações dentro do SpaceLab, o plano de testes completo para a qualificação do \gls{EPS2}, baseado nas diretrizes propostas, pode ser elaborado e executado, como forma de aplicação prática deste trabalho.
A elaboração de planos de teste de aceitação e \textit{proto-flight} também para este módulo poderá evidenciar as diferenças entre os objetivos de teste.

A extensão deste trabalho para os outros módulos de serviço de CubeSats seria de grande interesse, especialmente considerando que grande parte das discussões apresentadas poderão ser aplicadas diretamente para estes trabalhos.
Caberiam então, análises de topologias e arquiteturas dos respectivos módulos, aproveitando-se as considerações aqui apresentadas acerca das normas e a estrutura proposta para os planos de teste.

Sugere-se também a realização de estudos similares a este, com foco no processo de verificação, voltados à definição de requisitos e a e elaboração de planos de verificação. Apenas recentemente os processos de verificação e testes tem recebido mais atenção me missões de CubeSats, e trabalhos nesses tópicos contribuirão fortemente para o aumento da confiabilidade tanto dos módulos, quanto dos CubeSats como um todo.

% Também, um trabalho similar a este, porem focado ao processo de verificação como um todo seria de grande importância para missões de CubeSat, visto que apenas recentemente isto vem ganhando atenção e contribuiria para o aumento da confiabilidade tanto dos módulos, quanto dos CubeSats como um todo.