% ----------------------------------------------------------
\chapter{Desenvolvimento} \label{cap:desenvolvimento}
% ----------------------------------------------------------


Conforme \textcite{ecss-e-st-10-03}, o plano de testes é desenvolvido de acordo com o plano de verificação, que define quais dos requisitos relacionados ao produto serão testados.
O termo produto se refere ao item para o qual a norma está sendo aplicada.
A partir do plano de testes, documentos complementares são gerados, relacionados à especificação dos testes, procedimentos de teste e por fim relatórios de teste.

Além dos conceitos apresentados na \autoref{sec:normas-ecss}, estas normas contem uma série de definições e requisitos relacionados aos testes, além de linhas de base de testes a serem adotadas.
Para sua utilização é feito um processo de adequação, também chamado de tailoring, para cada produto, no caso deste trabalho, o \gls{EPS}.

Este processo de adequação inicia-se a partir de definições iniciais acerca do tipo de produto a qual o plano se aplicará e do tipo de modelos a serem utilizados, para então adaptarem-se as linha de base de testes e elaborar o plano de testes.

O documento de diretrizes proposto neste trabalho consistirá de um documento contendo um conjunto de orientações e direcionamentos, detalhando os principais aspectos e considerações relevantes acerca da adaptação destas normas para módulos \gls{EPS} de forma que, a partir dele, possa-se elaborar planos de teste para diferentes topologias e arquiteturas de \gls{EPS}.

% O documento proposto abordará os seguintes tópicos:
% \begin{alineas}
%     \item objetivo do plano de testes;
%     \item blocos de testes;
%     \item matriz de testes base;
%     \item documentação.
% \end{alineas}





% \red{
% Tópicos importantes:
% \begin{alineas}
%     \item Considerações iniciais:
%     \begin{alineas}
%         \item Verification level.
%         \item Verification stage.
%         \item Model philosophy.
%     \end{alineas}
%     \item Estrutura do plano de testes.
%     \begin{alineas}
%         \item Blocos de teste.
%         \item "Atividades" de teste.
%         \item Estrutura da documentação.
%     \end{alineas}
%     \item Matriz de testes base da ECSS.
%     \item Descrição de cada bloco de testes.
%     \begin{alineas}
%         \item Objetivo de cada bloco.
%         \item Testes necessários em cada bloco para um EPS.
%         \item Considerações adicionais.
%     \end{alineas}
%     \item Documentação.
% \end{alineas}
% }

% ----------------------------------------------------------
\section{Considerações Iniciais} \label{sec:consideracoes}
% ----------------------------------------------------------

De acordo com \textcite{ecss-e-st-10-02}, o processo de verificação é dividido em níveis de decomposição do sistema (\textit{verification levels}). Para cada nível, a verificação é feita múltiplos estágios (\textit{verification stages}), com objetivos específicos.
Testes se enquadram como um dos métodos utilizados dentro deste processo, inclusive sendo considerado o método que traz maior confiabilidade \cite{ecss-e-st-10-02}.

Desta forma, um plano de testes deve considerar o nível e estágio do processo de verificação no qual ele será executado.
Além disso, devem ser definidos os tipos de modelos físicos \textit{model philosophy} a serem utilizados nos testes.
O estágio de verificação, bem como o tipo de modelo adotado, definirão o objetivo principal do plano de testes elaborado.

A partir destas considerações, uma análise deve ser feita de modo que estes conceitos possam ser adaptados e aplicados a um módulo \gls{EPS} para CubeSats.


% ----------------------------------------------------------
\subsection{Nível de Verificação} \label{sec:nivel-verificacao}
% ----------------------------------------------------------


O nível de verificação (\textit{verification level}) está relacionado com a decomposição do sistema em diferentes níveis nos quais o processo de verificação é executado.
Na tabela do apêndice B.1 de ECSS-S-ST-00-01 \cite{ecss-s-st-00-01} mostra as divisões típicas, com exemplos.

Em ECSS-E-ST-10-03 \cite{ecss-e-st-10-03}, tratam-se dos testes para os níveis de \textit{space segment element} e \textit{space segment equipment}. O nível de \textit{space segment subsystem}, intermediário à estes, não é coberto nesta norma, de fato, é mencionado que para este nível, normalmente executam-se apenas testes funcionais.

Num primeiro momento, devido às nomenclaturas utilizadas e aos exemplos da tabela menciona, parece intuitivo atribuir os subsistemas de um CubeSat, e portanto o \gls{EPS}, ao nível de \textit{space segment subsystem}, porém uma análise mais minuciosa é necessária.

Primeiramente, visto que a plataforma de serviço é o principal foco de desenvolvimento do SpaceLab, é de grande interesse que sejam aplicados testes mais completos aos seus módulos, além de apenas testes funcionais básicos.
Além disso, analisando as definições presentes nas próprias normas da \gls{ECSS}, pode-se fazer argumento para a aplicação dos testes voltados para \textit{space segment element} aos módulos de serviço de CubeSat, mais especificamente, neste trabalho, ao\gls{EPS}.

Como visto na \autoref{sec:cubesats}, os módulos de serviço são responsáveis pelas funções fundamentais do nano satélite, e estão diretamente relacionados com o cumprimento de seus objetivos.
Observando a definição para o termo \textit{element}, nota-se que este está relacionado justamente com o cumprimento de um subconjunto dos objetivos do satélite.

\begin{citacao}
    element: combination of integrated equipment, components and parts. An element fulfils a major, self-contained, subset of a segment's objectives. \cite[p. 9]{ecss-s-st-00-01}.
\end{citacao}.

Além disso, nas próprias normas o termo \textit{service module} (módulo de serviço) é mencionado com \textit{element}. De fato, é mencionado tanto em \textcite{ecss-s-st-00-01} quanto em \textcite{ecss-e-st-10-03} que um elemento pode ser dividido em dois um mais elementos.

\begin{citacao}
    A space segment element can be composed of several space segment elements, e.g. a spacecraft is composed of instruments, a payload module and a service module. \cite[p. 10]{ecss-s-st-00-01}
\end{citacao}.

Outro ponto importante, esta divisão proposta nas normas está no contexto de um satélite de médio ou grande porte, com sistemas de complexidade muito maiores que um CubeSat, portanto, uma decomposição mais granulada dos sistemas é adequada neste aspecto.
Mesmo neste cenário, em \textcite{ecss-e-hb-10-02}, é mencionado a possibilidade de não se utilizar o nível de subsistema como forma de redução de custos.

Com isto, neste trabalho serão consideradas as recomendações da norma ECSS-E-ST-10-03 \cite{ecss-e-st-10-03} relacionadas à \textit{space segment element}.

% \red{
%     Referencias para aplicar "element" ao EPS:
%     \begin{alineas}
%         \item ECSS-E-ST-10-03 Section 6.1a menciona dividir os testes de um "elemet" em service module tests e payload module tests;
%         \item Tabela do apendice B.1 de ECSS-S-ST-00-01 mostra module como "element" porém também mostra "power" como "subsystem";
%         \item Definições de "component", "equipment", "subsystem" e "element" em ECSS-S-ST-00-01 Section 2.2;
%         \item ECSS-S-ST-00-01 Section 2.2.4 menciona service module como "element";
%         \item ECSS-E-HB-10-02 Section 5.2.1.3.2 menciona descartar o nível de subsystem como redução de custos;
%     \end{alineas}
% Equipment executa uma função específica, subsystem executa um conjunto de funções, element satisfaz um subset dos objetivos de um segment
% }

% \red{Apresentar os demais níveis de decomposição? Talvez adicionar um parágrafo relacionado na \autoref{sec:normas-ecss}.}


% ----------------------------------------------------------
\section{Documentação}
% ----------------------------------------------------------

Em relação à documentação relacionada ao processo de testes, a norma ECSS-E-ST-10-03 \cite{ecss-e-st-10-03} define quatro tipos de documentos a serem gerados, sendo eles: plano de teste (\textit{AIT Plan}), especificação de teste (\textit{Test Specification}), procedimento de teste (\textit{Test Procedure}) e relatório de teste (\textit{Test Report}).
As especificações detalhadas e requisitos acerca do conteúdo esperado de cada um destes documentos podem ser encontrados nos apêndices das normas ECSS-E-ST-10-03 \cite{ecss-e-st-10-03} e ECSS-E-ST-10-02 \cite{ecss-e-st-10-02}, a seguir será apresentada uma breve contextualização acerca do propósito de cada documento e por fim, considerações acerca da aplicação desta estrutura de documentação ao contexto deste trabalho.


% Test Plan
O plano de teste (\textit{AIT plan}) descreve todo o processo de testes, relacionando os testes planejados com os requisitos sendo verificados.
Ele contem o planejamento de atividades, matrizes de teste conectando os requisitos aos testes que os verificam, descrição dos equipamentos e instalações necessárias, documentação a ser produzida, bem como organização e cronograma.


% Test Specification
O documento de especificação de teste (\textit{Test Specification}) descreve em detalhes os requisitos e especificações do teste.
Ele define o propósito do teste, a abordagem utilizada, o item testado, os equipamentos utilizados, instrumentação e incertezas, condições do teste e tolerâncias, critérios de avaliação (sucesso e falha), documentação relacionada e cronograma.


% Test Procedure
O documento de procedimentos de teste (\textit{Test Procedure}) descreve os direcionamentos necessários para a execução do teste.
Neste documento são descritos os objetivos do teste, referência ao documento de especificação de teste correspondente, configuração do item testado, equipamentos necessários e instruções passo a passo para a execução do teste.

Em casos de testes mais simples, os documentos de especificação e procedimento de teste podem ser combinados em um único documento.
Em testes mais complexos, múltiplos documentos de procedimento de teste podem originar de um único documento de especificação.
Ainda, sobreposições de informação entre estes documentos devem ser minimizados, o foco do documento de especificação está na definição e requisitos do teste, enquanto que o documento de procedimentos tem foco operacional, com instruções passo a passo.


% Test report
O relatório de teste (\textit{Test Report}) descreve a execução do teste, os resultados, bem como avaliação e conclusões acerca dos requisitos do teste e critérios de avaliação.


Em relação à aplicabilidade deste tipo de documentação, algumas considerações podem ser feitas.
Primeiramente, ressalta-se que as definições destes documentos visam abranger testes altamente complexos para satélites de grande porte.
Outro ponto é que, seguindo à risca estas definições, uma grande quantidade de comentos é gerada, especialmente documentos de especificação e procedimentos de teste, visto que, a princípio, seriam gerados ao menos um conjunto destes documentos para cada teste.

Em uma missão de CubeSat, devido a simplicidade de projeto e também dos testes executados em relação à satélites de grande porte, bem como a proposta de baixo custo e rápido desenvolvimento, e especialmente considerando que neste trabalho o foco está em apenas um dos módulos do satélite, é possível simplificar e agrupar boa parte destes documentos.

Com isto em mente, é proposto que estes documentos sejam unificados e adaptados ao estilo de documentação de cada projeto.
No caso do SpaceLab, por exemplo, cada módulo possui um único documento descrevendo todas as características do mesmo, e incluindo também um planejamento de testes e relatórios de testes neste mesmo documento. Portanto, neste caso uma opção seria agrupar os quatro documentos em um único exemplar, e possivelmente anexa-lo ao documento principal do módulo.
Em projetos com múltiplos documentos, poderia-se manter documentos separados, mas agrupar cada tipo de documento em um único exemplar, por exemplo, agrupar todas as especificações de teste em um único documento de especificação.


% ----------------------------------------------------------
\subsection{Estrutura do Plano de Testes} \label{sec:estrutura-plano-testes}
% ----------------------------------------------------------


As definições das normas com relação ao documento de plano de teste (\textit{AIT Plan}) referem-se ao conteúdo esperado para o documento, mas não especifica diretamente a estrutura e organização deste documento.
Além disso, como mencionado acima, estas definições são passíveis de adaptação para o contexto de um módulo de CubeSat.
Sendo assim, a seguinte organização é proposta como ponto de partida para a elaboração do plano de testes:

\begin{alineas}
    \item introdução:
    \begin{alineas}
        \item contendo uma descrição dos objetivos e conteúdo do documento;
    \end{alineas}

    \item apresentação do produto:
    \begin{alineas}
        \item contento a descrição dos modelos físicos a serem utilizados e seu estado de desenvolvimento;
    \end{alineas}

    \item programa de testes:
    \begin{alineas}
        \item contento o planejamento e matriz de testes, referenciando cada teste com suas especificações, procedimentos e modelo utilizado;
        \item descrição dos blocos e atividades de teste;
        \item descrição do sequenciamento das atividades;
    \end{alineas}

    \item equipamentos e instalações:
    \begin{alineas}
        \item contendo uma relação dos equipamentos necessários e instalações a serem utilizadas;
    \end{alineas}

    \item documentação:
    \begin{alineas}
        \item contendo uma descrição dos documentos a serem gerados e seus conteúdos;
    \end{alineas}

    \item cronograma:
    \begin{alineas}
        \item caso o plano de testes tenha o intuito de ser recorrente, o cronograma não é necessário;
    \end{alineas}

\end{alineas}

Vale ressaltar que a estrutura acima é proposta como sugestão ou ponto de partida, e deve ser adaptada às necessidades e ao estilo de documentação de cada projeto.


% ----------------------------------------------------------
\section{Objetivos do Plano de Testes} \label{sec:objetivo-testes}
% ----------------------------------------------------------


Considerando o estágio de verificação e tipo de modelo adotado, o plano de testes é definido com um dos três objetivos principais: \textit{qualification testing}, \textit{acceptance testing}, \textit{proto-flight testing}.
A descrição e finalidade de cada objetivo foi apresentada na \autoref{sec:normas-ecss}.

A principal consequência de um determinado objetivo está na seleção dos testes a serem executados a partir das matrizes de teste base, e principalmente na intensidade e duração dos testes.

De fato, no documento ECSS-E-ST-10-03 \cite{ecss-e-st-10-03}, são apresentadas matrizes ou linhas de base de teste para cada objetivo, sendo que a diferença entre cada uma está na definição de quais testes são considerados como requeridos ou opcionais.
Em relação à intensidade e duração dos testes, são apresentadas tabelas com os dados específicos para cada objetivo.

Em relação às intensidades e durações, serão referenciadas estas tabelas diretamente no documento de diretrizes quando necessário.






% ----------------------------------------------------------
\section{Linha de Base de Testes} \label{sec:linhas-base}
% ----------------------------------------------------------

As linhas de base de testes apresentadas em \textcite{ecss-e-st-10-03}, assim como o restante da norma, foram elaboradas de forma bastante abrangente e direcionadas à satélites de grande porte.
Sendo assim, considerando a simplicidade de projeto de um CubeSat e o contexto de uma missão universitária, uma versão adaptada destas matrizes de base será adotada.

De fato, será proposta uma única matriz para os três diferentes objetivos, porém esta será elaborada de forma a abranger diferentes topologias e arquiteturas.
Ainda, visto que o documento de diretrizes é voltado diretamente para módulos \gls{EPS}, é possível propor testes mais específicos, especialmente no caso de testes funcionais e de missão, proporcionando um maior direcionamento em relação as matrizes base.

% Considerações sobre testes em missões anteriores

Com isso, a partir das linhas de base de testes apresentadas na norma ECSS-E-ST-10-03 \cite{ecss-e-st-10-03}, considerou-se que a matriz de testes a ser proposta deve conter os seguintes testes:

\begin{alineas}
    \item testes funcionais;
    \item testes de performance;
    \item testes de missão;
    \item propriedades físicas;
    \item vibração;
    \item termo vácuo.
\end{alineas}


% Considerações a respeito de cada item


Para os testes de propriedades físicas, vibração e termo vácuo (testes mecânicos), os requisitos de intensidade e duração, bem como detalhes acerca da especificação destes testes podem ser referenciados diretamente das seções 6.5.2 e 6.5.4, e das tabelas 6-2, 6-4 e 6-6 da norma ECSS-E-ST-10-03 \cite{ecss-e-st-10-03}.

Em relação aos testes funcionais, de performance e de missão, apenas uma visão geral da finalidade de cada um destes testes é apresentada, visto que estes tipos de teste são bastante específicos ao produto sendo testado e à própria missão, e devem ser especificados caso a caso.

O documento de diretrizes tem o intuito de ser aplicável às diferentes implementações de módulos, sendo assim, os testes funcionais, de performance e de missão devem levar um consideração as diferentes possibilidades.
No \autoref{cap:fundamentacao}, foram apresentadas diversas topologias e arquiteturas de \gls{EPS}s, bem como campanhas de testes destes módulos,que serão analisadas a fim de propor uma matriz de testes mais abrangente.


% ----------------------------------------------------------
\section{Testes Funcionais, de Performance e de Missão} \label{sec:analises-testes}
% ----------------------------------------------------------

Conforme apresentados na norma ECSS-E-ST-10-03 \cite{ecss-e-st-10-03}, os testes funcionais, de performance e de missão tem as seguintes finalidades:

\begin{alineas}
    \item testes funcionais tem a finalidade de verificar que o produto apresenta o funcionamento adequado de acordo com as suas especificações e requisitos em todos os seus modos de operação:
    \begin{alineas}
        \item pode-se ainda dividir-se em testes funcionais mecânicos e elétricos;
    \end{alineas}
    \item testes de performance tem a finalidade de verificar que o produto apresenta a performance adequada na execução de suas funções de acordo com as suas especificações e requisitos;
    \item testes de missão tem a finalidade de simular casos de missão esperados, em cenários nominais e críticos, durante a operação da missão, dentro do que se pode similar em terra:
    \begin{alineas}
        \item casos de missão referem-se a situações e eventos esperados durante a missão, bem como a operação nominal de acordo com o planejamento da missão.
    \end{alineas}
\end{alineas}

Com isso em mente, analisando as topologias e arquiteturas de diferentes projetos, apresentadas na \autoref{sec:arq-top}, identificou-se um conjunto de características presentes nos diferentes designs relacionadas à funcionalidades de um módulo \gls{EPS}:

\begin{alineas}
    \item múltiplos estágios de conversão:
    \begin{alineas}
        % \item reguladores para \gls{MPPT}; % Interface com painéis solares
        % \item reguladores para bateria (\gls{BCR});
        % \item reguladores para os barramentos de saída;
        \item os módulos possuem reguladores em diversos pontos da distribuição de energia: nas interfaces com os painéis solares, para regulagem de carga da bateria e para os barramentos de saída;
    \end{alineas}

    \item controle dos barramentos:
    \begin{alineas}
        \item os barramentos de saída costumam ter chaves de potência possibilitando controle individual;
        \item pode-se ter também o controle via pinos de \textit{enable} dos próprios reguladores;
    \end{alineas}

    \item circuitos de proteção;
    \begin{alineas}
        % \item proteção dos barramentos;
        % \item proteção das baterias;
        \item estas mesmas chaves de potência costumam possuir limitação de corrente como proteção;
        \item os módulos podem possuir também circuitos de proteção para as baterias, utilizando chaves para controle das direções de carga e descarga;
    \end{alineas}

    \item aquecimento de baterias;
    \begin{alineas}
        \item o sistema de aquecimento normalmente é feito através de resistores de potência;
    \end{alineas}
    
    \item monitoramento;
    \begin{alineas}
        % \item tensão e corrente nos barramentos;
        % \item tensão e corrente nas baterias;
        % \item temperatura das baterias;
        \item sensores de tensão e corrente são empregados nos barramentos principais e em diversos pontos de interesse;
        \item sensores de temperatura também são utilizados, especialmente nas baterias;
    \end{alineas}

    \item presença de microcontrolador;
    \begin{alineas}
        % \item firmware;
        % \item protocolos de comunicação;
        % \item leitura/escrita de dados;
        % \item funções específicas; % startup do satélite, modos de energia, modos de operação, proteções implementadas via software...
        \item microcontroladores são utilizados para comunicação, leitura dos sensores e gerenciamento de dados;
        \item em alguns casos, são utilizados também para execução de algoritmos, como \gls{PO} para controle do \gls{MPPT};
        \item podem executar também funções operacionais da missão, como inicialização do satélite e controle dos modos de operação.
    \end{alineas}
\end{alineas}

Todo conjunto de características servirá de base para os testes funcionais propostos na matriz do documento de diretrizes, de forma que, ao ser elaborado um plano de testes específico, possam ser selecionados os testes funcionais relevantes para uma dada arquitetura.

Como base para os testes de performance e de missão, convém analisar-se as campanhas de testes de módulos \gls{EPS} apresentadas na \autoref{sec:testes-epss}.
Na \autoref{tab:testes-eps}, tem-se uma visão geral dos principais tipos de testes executados nos módulos em cada campanha de testes apresentada.
Uma descrição mais detalhada sobre os testes em cada módulo encontra-se na seção menciona acima.

\begin{table}[ht!]
    \ABNTEXfontereduzida
    \centering
    \caption{Tipos de testes executados em campanha de testes de EPSs}
    \begin{tabular}{ccccccc}
        \toprule
        \multirow{2}{*}{\textbf{EPS}} & \multicolumn{6}{c}{\textbf{Testes}} \\
        \cline{2-7}
        & \textbf{Inspeções} & \textbf{Funcionais} & \textbf{Eficiência} & \textbf{Missão} & \textbf{Estresse} & \textbf{Ambiente} \\
        \midrule
        \midrule
        Aalto-2   & - & X & X & - & - & - \\
        ESTCube-1 & - & X & X & - & X & X \\
        MIST      & - & X & X & X & - & - \\
        EPS 2.0   & X & X & - & - & - & - \\
        \bottomrule
    \end{tabular}
    \label{tab:testes-eps}
\end{table}


Aqui, inspeções referem-se principalmente em inspecionar visualmente as condições de fabricação e transporte dos módulos, físicos utilizados.
Testes de ambiente referem-se aos testes de vibração, termo vácuo, choque mecânico, entre outros, ou seja, testes relacionados às condições do ambiente espacial e do lançamento.
Testes de estresse referem-se a submeter os componentes aos seus limites máximos e também testes como o burn-in.
Os testes de eficiência referem-se principalmente à eficiência dos diversos conversores presentes nos módulos.

Testes classificados como funcionais envolvem teste dos conversores com as cargas esperadas, testes de comunicação, leitura de sensores, funções do microcontrolador, funcionamento de algoritmos, entre outros.

Como pode ser visto, dentre os módulos analisado, um grande foco é dado para testes funcionais, principalmente para o funcionamento dos conversores e circuitos de proteção, porém, apenas no EPS 2.0 foram relatados testes funcionais além destes.
Nota-se também que em nenhuma campanha foram realizados testes de todos os tipos listados.

Os testes de eficiência dos conversores são o principal interesse relacionado à performance nos \gls{EPS}s.
Em alguns casos, tem-se inclusive requisitos específicos relacionados à eficiência, como em \textcite{aalto-eps} em que foi dado um grande foco em testes de eficiência de conversores.

Em relação aos testes de missão, apenas em \textcite{mist-eps} foram relatados simulações de diferentes cenários esperados.
Neste caso, foram simulados o funcionamento das diferente cargas úteis do satélite, em termos de tempo de acionamento e consumo, conforme a operação planejada e em diferentes cenários.


% ----------------------------------------------------------
\section{Matriz de Testes Base} \label{sec:matriz-base}
% ----------------------------------------------------------


Com base nas análises das diferentes topologias e arquiteturas, bem como das campanhas de testes e também das normas da \gls{ECSS}, é proposta a matriz de testes mostrada na \autoref{tab:matriz-base} com linha de base.

\begin{table}
    \ABNTEXfontereduzida
    \centering
    \caption{Matriz de testes base}
    \begin{tabular}{cl}
        \toprule
        \textbf{Test Block} & \textbf{Test Activity} \\
        \midrule
        \midrule
        \multirow{4}{*}{Inspeções}      & Inspeção de Fabricação                \\
                                        & Inspeção Elétrica                     \\
                                        & Inspeção Mecânica                     \\
                                        & Inspeção de Integração                \\
        \midrule
        \multirow{7}{*}{Funcionais}     & Sistema de \textit{harvesting}        \\
                                        & Reguladores dos canais de saída       \\
                                        & Gerenciamento das Baterias    \\
                                        & Chaveamento dos canais de saída       \\
                                        & Circuitos de proteção                 \\
                                        & Leitura de sensores                   \\
                                        & Barramentos de comunicação            \\
        \midrule
        \multirow{5}{*}{Performance}    & Consumo de potência do módulo         \\
                                        & Eficiência do regulador do sistema de \textit{harvesting} \\
                                        & Eficiência dos reguladores dos canais de saída \\
                                        & Eficiência do regulador de carga das baterias \\
                                        & Eficiência do sistema                 \\
        \midrule
        Missão                          & Casos de missão                       \\
        \midrule
        \multirow{4}{*}{Ambientais}     & Vibração                              \\
                                        & Termo vácuo                           \\
                                        & Ciclagem térmica                      \\
                                        & Bake-out                              \\
        \bottomrule
    \end{tabular}
    \fonte{Elaborado pelo autor.}
    \label{tab:matriz-base}
\end{table}


A matriz está organizada em blocos teste, sendo estes: integração, testes funcionais, testes de performance, testes de missão e testes ambientais.
Cada bloco é composto por um conjunto de atividades de testes relacionadas, por fim, cada atividade é composta por um conjunto de testes individuais.
Ainda, a matriz deve relacionar cada teste individual com suas respectivas especificações, procedimentos, modelo físico utilizado e requisitos sendo verificados, de fora que, no plano de testes específico, esta matriz apresentaria colunas adicionais relacionadas a estes pontos, podendo inclusive ser dividida em múltiplas matrizes para melhor organização.
Desta forma,testes relacionados podem ser agrupados e executados de maneira organizada.

A nível de testes individuais, a dependência da arquitetura específica do módulo alvo dos testes aumenta significativamente, tornando inviável a proposta de testes individuais que possam abranger diferentes arquiteturas.
Com isso, limitou-se ao nível de atividades de teste para a matriz proposta.

Ressalta-se também que esta matriz é proposta como uma linha de base, e testes adicionais específicos a uma determinada arquitetura, não cobertos na matriz base, podem ser adicionados conforme necessário.

O bloco de inspeção tem como finalidade verificar a integridade do processo de fabricação e conformidade do modelo físico em relação aos arquivos de projeto, garantindo que não hajam defeitos de fabricação nos modelos testados.
Este bloco é composto pelas seguintes atividades:

\begin{alineas}
    \item inspeção de fabricação:
    \begin{alineas}
        \item tem a finalidade de verificar a integridade do processo de fabricação e transporte;
        \item consiste de inspeções visuais das condições de embalagem e transporte, bem como conformidade com requisitos do processo de fabricação;
    \end{alineas}

    \item inspeção elétrica:
    \begin{alineas}
        \item tem a finalidade de verificar a integridade elétrica do módulo;
        \item consiste de verificar conformidade em relação aos esquemáticos, qualidade das soldas, ausência de curto circuitos e alimentar o módulo pela primeira vez;
    \end{alineas}

    \item inspeção mecânica:
    \begin{alineas}
        \item tem a finalidade de verificar as propriedade físicas em relação aos arquivos de projeto;
        \item consiste na medição das dimensões da placa, massa, posição e tamanho das furações (\textit{mounting holes});
    \end{alineas}

    \item inspeção de integração:
    \begin{alineas}
        \item tem a finalidade de verificar que o módulo pode ser fisicamente integrado ao satélite;
        \item consiste em verificar a pinagem e posição de conectores em relação aos esquemáticos;
    \end{alineas}
\end{alineas}


O bloco de testes funcionais tem como finalidade verificar que o módulo é capaz de executar todas as suas funções requeridas.
Este bloco é composto pelas seguintes atividades:

\begin{alineas}
    \item sistema de harvesting:
    \begin{alineas}
        \item tem a finalidade de verificar o funcionamento do sistema de harvesting do módulo;
        \item consiste em teste dos conversores utilizados, bem como o funcionamento de sistemas de \gls{MPPT} e algoritmos relacionados;
    \end{alineas}

    \item canais de saída:
    \begin{alineas}
        \item tem a finalidade de verificar o funcionamento dos conversores dos barramentos de saída do módulo;
        \item consiste na aplicação de cargas aos conversores, conforme os limites esperados durante a operação da missão;
    \end{alineas}

    \item consumo de potência:
    \begin{alineas}
        \item tem a finalidade de verificar o consumo de potência do módulo isoladamente;
        \item consiste na medição do consumo do módulo em estado ocioso, sem cargas conectadas.
    \end{alineas}

    \item gerenciamento das baterias:
    \begin{alineas}
        \item tem a finalidade de verificar o funcionamento do sistema de gerenciamento e monitoramento da bateria;
        \item consiste no teste de reguladores de carga, funcionamento de sistemas de monitoramento, testes do sistema de aquecimento e algoritmos de controle relacionados;
    \end{alineas}

    \item chaveamento dos canais de saída:
    \begin{alineas}
        \item tem a finalidade de verificar o funcionamento do sistema de controle dor barramentos de saída;
        \item consiste em testes das chaves de potências dos barramentos e funcionamento dos pinos de \textit{enable} do s conversores;
    \end{alineas}

    \item circuitos de proteção:
    \begin{alineas}
        \item tem a finalidade de verificar o funcionamento dos circuitos de proteção presentes no módulo;
        \item consiste no teste da limitação de corrente de chaves de potência, controle de carga e descarga das baterias e algoritmos relacionados;
    \end{alineas}

    \item leitura de sensores:
    \begin{alineas}
        \item tem a finalidade de verificar o funcionamento e a leitura correta dos sensores presentes no módulo;
        \item consiste no testes e comparação das leituras dos sensores do módulo com instrumentos de medição;
    \end{alineas}

    \item barramentos de comunicação:
    \begin{alineas}
        \item tem a finalidade de verificar a integridade e correta operação dos barramentos de comunicação do módulo;
        \item consiste em testes dos sistemas de comunicação tanto com outros módulos do satélite quatro com periféricos do próprio módulo, verificar a integridade dos sinais e da informação transmitida.
    \end{alineas}
\end{alineas}

O bloco de testes de performance tem como finalidade de verificar que a performance do módulo na execução de suas funções está de acordo com os requisitos.
O grande foco deste bloco está em verificar a eficiência dos diversos estágios de conversão presentes no \gls{EPS}, o que costuma ser feito aplicando-se cargas incrementais aos reguladores, obtendo-se a eficiência para diversos pontos de operação.
Este bloco é composto pelas seguintes atividades:

\begin{alineas}
    \item eficiência do sistema de harvesting:
    \begin{alineas}
        \item tem a finalidade de verificar a eficiência dos reguladores utilizados no sistema de harvesting;
    \end{alineas}

    \item eficiência dos conversores de saída:
    \begin{alineas}
        \item tem a finalidade de verificar a eficiência dos reguladores utilizados nos barramentos de saída do módulo;
    \end{alineas}

    \item eficiência dos reguladores de carga das baterias:
    \begin{alineas}
        \item tem a finalidade de verificar a eficiência dos reguladores utilizados na interface com as baterias;
    \end{alineas}

    \item eficiência do sistema:
    \begin{alineas}
        \item tem a finalidade de verificar a eficiência do sistema como um todo;
    \end{alineas}

\end{alineas}


O bloco de testes de missão tem como finalidade verificar a operação correta do módulo em relação ao conceito de operação da missão.
As atividade de teste deste envolverão a simulação, dentro do que é factível em terra, de cenários esperados durante a operação do módulo, em concordância com o conceito de operações da missão.
Incluem-se também simulações cenários críticos e situações adversas, bem como casos com requisitos de tempo críticos (\textit{time critical}).
No caso deste bloco, mesmo a nível de atividades de teste, há uma grande dependência em relação à missão e aos objetivos específicos de cada implementação, portanto não serão propostas atividades ou casos específicos neste momento.

O bloco de testes ambientais tem a finalidade de verificar que o módulo é capaz de sobreviver e operar nas condições ambientais em que será submetido.
As atividades envolvem testes de vibração, que remetem às condições durante o lançamento, e termo vácuo, que remete ao ambiente espacial.


% ----------------------------------------------------------
\section{Requisitos de Teste}
% ----------------------------------------------------------

\red{Comentar sobre requisitos relacionados à condições de teste, tolerancias, incertezas e etc. No caso, serão referenciadas diretamente as normas no documento como recomendações, visto que aplica-los à risca em uma missão universitária seria demasiado demorado e custoso (basicamente impossivel)}

As normas da \gls{ECSS} apresentam também requisitos relacionados à testes de forma geral, que independem do produto para qual está sendo realizado o tailoring.
Estes requisitos abordam condições de testes, especificação dos locais de teste, equipamentos utilizados, tolerâncias e incertezas, revisões, entre outros.

Nestes casos, quando necessário, serão referenciadas as normas no documento de diretrizes diretamente.

\red{Comentar sobre requisito de rastreabilidade dos testes até os requisitos verificados}
