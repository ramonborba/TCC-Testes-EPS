% ----------------------------------------------------------
\chapter{Plano de Testes Base}
% ----------------------------------------------------------

Com base nos estudos e analises realizadas, foi elaborado um documento, nomeado \textit{EPS Test Plan Guidelines} contendo diretrizes e recomendações acerca da elaboração de planos de teste para módulos \gls{EPS} no contexto de missões de CubeSats.

Visto que a utilização primária será para os módulos desenvolvidos no SpaceLab, e também motivação da realização do trabalho veio das experiências adquiridas na participação dos projetos do laboratório, foi utilizado o modelo e estilo de documentação utilizados no SpaceLab, no idioma Inglês.
No \autoref{app:plano-testes-base} pode ser encontrada a primeira versão deste documento.

O documento cobre os tópicos discutidos no \autoref{cap:desenvolvimento}, e está organizado da seguinte maneira:

\begin{alineas}
    \item introdução:
    \begin{alineas}
        \item contendo uma descrição do propósito do documento;
    \end{alineas}

    \item documentação:
    \begin{alineas}
        \item contendo uma estrutura base para o plano de testes, bem como uma descrição da documentação associada ao plano;
    \end{alineas}

    \item objetivos de teste:
    \begin{alineas}
        \item contendo uma descrição dos principais objetivos de um plano de testes.
    \end{alineas}

    \item requisitos gerais:
    \begin{alineas}
        \item contendo referência aos requisitos gerais de teste definidos nas normas;
    \end{alineas}

    \item matriz de testes base:
    \begin{alineas}
        \item contendo a matriz de testes base proposta, bem como uma descrição dos principais blocos e atividades de teste.
    \end{alineas}

\end{alineas}


Apesar de que a aplicação primária será aos módulos \gls{EPS} do SpaceLab, o documento foi elaborado de forma que possa ser utilizado por outros grupos, e para diferentes módulos \gls{EPS}, e estará em constante desenvolvimento e aprimoramento.