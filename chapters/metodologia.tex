% ----------------------------------------------------------
\chapter{Metodologia}
% ----------------------------------------------------------

O documento de diretrizes proposto neste trabalho terá como principal base as normas da \gls{ECSS}, especialmente nos documentos referentes a testes \cite{ecss-e-st-10-03} e verificação \cite{ecss-e-st-10-02}, adaptadas ao cenário de um módulo de serviço de um CubeSat.

Uma análise mais detalhada destes documentos será realizada, de forma a identificar os principais aspectos e considerações relevantes ao adotar-se estas normas em uma missão universitária.

A partir desta análise e dos conceitos apresentados na \autoref{sec:normas-ecss}, as diretrizes acerca da estrutura e organização do plano de testes, bem como da documentação relacionada, poderão ser definidas.
Também, as linhas de base de testes apresentadas nas normas poderão ser adequadas ao módulo \gls{EPS}.

% Com o intuito de propor uma estrutura organizada e completa para o plano de testes, as normas e padrões da \gls{ECSS} serão tomados como referencial.
% Será feita uma análise mais detalhada dos documentos referentes a testes \cite{ecss-e-st-10-03} e verificação \cite{ecss-e-st-10-02}, identificando-se os aspectos principais e de maior relevância considerando o cenário de uma missão universitária.

Para que, no documento de diretrizes, possa-se propor uma matriz de testes base direcionada a módulos \gls{EPS}, serão realizadas análises das topologias, arquiteturas e campanhas de testes apresentadas nas Seções \ref{sec:topologias}, \ref{sec:arquiteturas} e \ref{sec:testes-epss}.

Destas análises, serão identificadas as principais características e funcionalidades destes módulos, assim como os principais testes executados nas campanhas de missões anteriores. Estas informações servirão de referência para a definição das atividades de teste a serem incluídas na matriz.
Assim, será possível levantar uma linha de base de testes direcionada a \gls{EPS}s, mas ainda de forma a abranger diferentes topologias e arquiteturas.

% A fim de propor uma matriz de testes base, adaptável a diferentes módulos, as topologias e arquiteturas de \gls{EPS} apresentadas, bem como as campanhas de testes de missões prévias, serão analisadas e consideradas. Identificando-se os diferente aspectos e funcionalidades, assim como os diferentes testes executados, blocos de testes mais completos poderão ser propostos.

A participação em projetos como FloripaSat-1, GOLDS-UFSC e Constelação Catarina, os anos de experiência no desenvolvimento e testes de módulos \gls{EPS}, bem como o convívio diário com professores e colegas de laboratório no ambiente do SpaceLab agregam um conhecimento prévio significativo acerca dos aspectos e do cenário do desenvolvimento de missões espaciais universitárias.
As considerações e adaptações adotadas acerca da aplicabilidade de certos aspectos das normas são resultado destas experiências.