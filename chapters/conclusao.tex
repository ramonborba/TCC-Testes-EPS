% ----------------------------------------------------------
\chapter{Conclusão}
% ----------------------------------------------------------

Neste trabalho, foram realizadas análises acerca de deferentes topologias e arquiteturas de módulos de gerenciamento de energia de CubeSats, bem como de suas respectivas campanhas de testes. Foram analisadas também as normas da \gls{ECSS} relacionadas à testes e verificação, considerando o contexto de um módulo de CubeSat. Com isto foram identificadas as principais funcionalidades presentes nas diferentes arquiteturas, bem como os principais tipos de teste executados. Ainda, foram identificados os principais aspectos das normas em relação à elaboração de planos de teste, bem como discutidas as principais considerações a serem feitas ao aplicar-se estas normas a um módulo de CubeSat.

A partir destas análises, foi elaborado o documento \textit{EPS Test Plan Guidelines} contendo diretrizes e orientações acerca da elaboração de planos de testes para módulos \gls{EPS} de CubeSats, baseado nas normas da \gls{ECSS}, de forma que seja aplicável à diferentes topologias e arquiteturas.

Por fim, como forma de demonstração, um plano de testes para o módulo de gerenciamento de energia \gls{EPS2}, desenvolvido no SpaceLab, foi proposto, baseado nas diretrizes e orientações desenvolvidas neste trabalho.

\section{Trabalhos Futuros}

Como sugestão para trabalhos futuros, a análise de um número maior de arquiteturas, bem como de campanhas de teste, contribuiria para a elaboração de uma matriz de testes base ainda mais abrangente.

Também, um trabalho similar a este, porem focado ao processo de verificação como um todo seria de grande importância para missões de CubeSat, visto que apenas recentemente isto vem ganhando atenção e contribuiria para o aumento da confiabilidade tanto dos módulos, quanto dos CubeSats como um todo.