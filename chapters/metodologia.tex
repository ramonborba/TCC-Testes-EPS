% ----------------------------------------------------------
\chapter{Metodologia}
% ----------------------------------------------------------

% Adicionar que o trabalho foi desenvolvido pensando no spacelab como aplicação primária e como motivação

Com o intuito de propor uma estrutura organizada e completa para o plano de testes, as normas e padrões da \gls{ECSS} serão tomados como referencial.
Será feita uma análise mais detalhada dos documentos referentes a testes \cite{ecss-e-st-10-03} e verificação \cite{ecss-e-st-10-02}, identificando-se os aspectos principais e de maior relevância considerando o cenário de uma missão universitária.

A fim de propor uma matriz de testes base, adaptável a diferentes módulos, as topologias e arquiteturas de \gls{EPS} apresentadas, bem como as campanhas de testes de missões prévias, serão analisadas e consideradas. Identificando-se os diferente aspectos e funcionalidades, assim como os diferentes testes executados, blocos de testes mais completos poderão ser propostos.

A participação em projetos como FloripaSat-1, GOLDS-UFSC e Constelação Catarina, os anos de experiência no desenvolvimento e testes de módulos \gls{EPS}, bem como o convívio diário com professores e colegas de laboratório no ambiente do SpaceLab agregam um conhecimento prévio significativo acerca dos aspectos e do cenário do desenvolvimento de missões espaciais universitárias.
As considerações e adaptações adotadas acerca da aplicabilidade de certos aspectos das normas são resultado destas experiências.