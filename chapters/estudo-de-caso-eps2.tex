% ----------------------------------------------------------
\chapter{Estudo de Caso - EPS 2.0}
% ----------------------------------------------------------

A partir das diretrizes e orientações do documento \textit{EPS Test Plan Guidelines}, um plano de testes para o módulo \gls{EPS2}, desenvolvido no SpaceLab, foi elaborado.
No \autoref{app:plano-testes-eps2} pode ser encontrada a primeira versão deste documento.

O testes foram selecionados considerando a execução nas instalações do SpaceLab, levando em conta os equipamentos e infraestrutura disponíveis no mesmo. Por este motivo, testes ambientais não foram incluídos.

Ainda, visto que um dos interesses do laboratório é de viabilizar a comparação de performance entre os diferentes módulos de \gls{EPS} desenvolvidos, um requisito adicional representando este propósito, foi adicionado ao plano, além dos requisitos da missão deste módulo.

O documento foi estruturado da seguinte maneira:


\begin{alineas}
    \item introdução:
    \begin{alineas}
        \item contendo uma descrição do propósito do plano de testes, bem como considerações acerca de sua elaboração;
    \end{alineas}

    \item modelos de hardware:
    \begin{alineas}
        \item contendo uma descrição dos modelos físicos a serem utilizados para os testes e seu estado de construção;
    \end{alineas}

    \item requisitos a serem verificados:
    \begin{alineas}
        \item contendo uma listagem dos requisitos da missão verificados por este plano de testes.
    \end{alineas}

    \item programa de testes:
    \begin{alineas}
        \item contendo a matriz de testes proposta, descrição dos principais blocos e atividades de teste e fluxo de execução.
    \end{alineas}

    \item instalações de teste:
    \begin{alineas}
        \item contendo uma descrição das instalações a serem utilizadas e listagem dos equipamentos disponíveis;
    \end{alineas}

    \item documentação:
    \begin{alineas}
        \item contendo uma descrição dos documentos a serem elaborados a partir deste plano de testes;
    \end{alineas}

\end{alineas}


\section{Discussão}

O planejamento de testes inicial do \gls{EPS2}, apresentado na \autoref{sec:arq-top}, apesar de apresentar uma organização dos testes similar à proposta neste trabalho, com uma matriz de teste e um divisão em tipos de teste, possuía diversos pontos passíveis de melhoria.

Em relação à estrutura e documentação, o planejamento apresentava apenas uma breve descrição dos testes, sem menção aos requisitos ou enquadramento em um dos objetivos de teste apresentados na \autoref{sec:normas-ecss}.

Em relação aos testes propostos, este foi o único planejamento à apresentar inspeções dentre as campanhas de teste analisada, porém além destas, limitou-se a testes funcionais.
Ainda, os testes funcionais propostos focam quase exclusivamente em funcionamento do firmware do módulo.


Com a aplicação das diretrizes e recomendações desenvolvidas neste trabalho, o novo plano de testes proposto apresenta uma melhor organização em relação aos objetivos de teste,  documentação e relação com os requisitos verificados.

Além disso, os testes e inspeções ja existentes foram reorganizados e testes funcionais, de performance e de missão foram propostos, tornando o plano mais completo.