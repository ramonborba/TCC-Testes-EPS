% ----------------------------------------------------------
\chapter{Desenvolvimento}
% ----------------------------------------------------------

\red{Descrever as considerações e raciocínios para a elaboração do plano baseado no que foi apresentado na fundamentação}

\red{Descrever a estrutura e organização do plano de testes.}

\red{
Tópicos importantes:
\begin{itemize}
    \item Considerações iniciais:
    \begin{itemize}
        \item Verification level.
        \item Verification stage.
        \item Model philosophy.
    \end{itemize}
    \item Estrutura do plano de testes.
    \begin{itemize}
        \item Blocos de teste.
        \item "Atividades" de teste.
        \item Estrutura da documentação.
    \end{itemize}
    \item Matriz de testes base da ECSS.
    \item Descrição de cada bloco de testes.
    \begin{itemize}
        \item Objetivo de cada bloco.
        \item Testes necessários em cada bloco para um EPS.
        \item Considerações adicionais.
    \end{itemize}
    \item Documentação.
\end{itemize}
}


De acordo com \textcite{ecss-e-st-10-02}, o processo de verificação é dividido em níveis de decomposição do sistema (\textit{verification levels}). Para cada nível, a verificação é feita multiplos estagíos (\textit{verification stages}), com objetivos específicos.
\red{Acresentar um ou dois parágrafos sobre verificação na \autoref{sec:normas-ecss}, importante ressaltar os estágios aplicaveisa à uma missão de CubeSat}

Dentro deste processo, testes se enquadram como um dos métodos utilizados para verificação de requisitos, inclusive sendo considerado o método que traz maior confiabilidade.
Um plano de testes deve ser preparado para cada nível e estágio do processo de verificação.

Para a verificação por meio de testes, devem ser definidos os tipos de modelos que serão utilizados (\textit{model philosophy}).

Com isso, para a elaboração de um plano de testes, os seguintes aspectos iniciais, considerando um módulo \gls{EPS}, precisam ser definidos:
\begin{itemize}
    \item \textit{Verification level};
    \item \textit{Verification stage};
    \item \textit{Model philosophy}.
\end{itemize}

Em relação ao nível de decomposição do sistema, no documento ECSS-E-ST-10-03, são abordados os requisitos de teste para os níveis de \textit{space segment element} e \textit{space segment equipment}.
Observando a tabela mostrada no Apêndice B.1 de \textcite{ecss-s-st-00-01}, bem como os termos e definições apresentados na seção 2.2 do mesmo documento, no contexto de um CubeSat, considerou-se o \gls{EPS} como um \textit{space segment element}.

\red{Space segment subsystem, de acordo com as normas, passam apenas por testes funcionais, porém, há argumentos ára se aplicar os testes de space segment element ao EPS. Apesar de, na clssificação da tabela, o EPS se encaixar em subsistema, em um CubeSat, cada subsistema tem uma relevância muito maior. Devido a simplicidade de projeto, decomposição de um CubeSat tem menos níveis, e neste contexto, pode-se considerar que um subsistema teria a mesma relevância atribuída aos elements.
Outro ponto é que no spacelab, o desenvolvimento da plataforma de serviço é o principal foco, há um grande interesse em realizar este tipo de testes nos módulos individualmente, suprindo ess gap que existe na ECSS.
}

\red{
    Referencias para aplicar "element" ao EPS:
    \begin{itemize}
        \item ECSS-E-ST-10-03 Section 6.1a menciona dividir os testes de um "elemet" em service module tests e payload module tests;
        \item Tabela do apendice B.1 de ECSS-S-ST-00-01 mostra module como "element" porém também mostra "power" como "subsystem";
        \item Definições de "component", "equipment", "subsystem" e "element" em ECSS-S-ST-00-01 Section 2.2;
        \item ECSS-S-ST-00-01 Section 2.2.4 menciona service module como "element";
        \item ECSS-E-HB-10-02 Section 5.2.1.3.2 menciona descartar o nível de subsystem como redução de custos;
    \end{itemize}
Equipment executa uma função específica, subsystem executa um conjunto de funções, element satisfaz um subset dos objetivos de um segment
}

\red{Apresentar os demais níveis de decomposição? Talvez adicionar um parágrafo relacionado na \autoref{sec:normas-ecss}.}

Os estágios de verificação estão diretamente relacionados aos objetivos do plano de testes, sendo os princpais estágios: \textit{qualification}, \textit{acceptance}, \textit{pre-launch}, \textit{in-orbit} e \textit{post-landing}.


% Considerando que CubeSats normalmente não são recuperados após o fim da missão, o estágio de \textit{post-landing} não se aplica.